\chapter{分布式人工智能与Agent(真体)}

\begin{question}
分布式人工智能系统有何特点?试与多真体系统的特性加以比较。  
\end{question}
\begin{solution}
\end{solution}

\begin{question}
什么是真体?你对agent的定义译法有何见解?
\end{question}
\begin{solution}
Agent是能够通过传感器感知其环境,并借助执行器作用于该环境的实体,可看作是从感知序列到动作序列的映射。\par
其特性为:行为自主性,作用交互性,环境协调性,面向目标性,存在社会性,工作协作性,运行持续性,系统适应性,结构分布性,功能智能性。\par  
把agent译为艾真体的原因主要有:
	\begin{enumerate}
	\item 一种普遍的观点认为,agent是一种通过传感器感知其环境,并通过执行器作用于该环境的实体;
	\item ``主体''一词考虑到了agent具有自主性,但并未考虑agent还具有交互性,协调性,社会性,适应性和分布性的特性;
	\item ``代理''一词在汉语中已经有明确的含义,并不能表示出agent的原义;
	\item 把agent译为``真体'',含有一定的物理意义,即某种``真体''或事物,能够在十分广泛的领域内得到认可;
	\item 在找不到一个确切和公认的译法时,宜采用音译。
	\end{enumerate}
\end{solution}

\begin{question}
真体在结构上有何特点?在结构上又是如何分类的?每种结构的特点为何?  
\end{question}
\begin{solution}
\end{solution}

\begin{question}
真体为什么需要互相通信?
\end{question}
\begin{solution}
一些交谈能向受话者传送信息,还有一些交谈要受话者采取行动。通信的双重目的就是建立信任和创建社会联系。  
\end{solution}

\begin{question}
试述真体通信的步骤、类型和方式。
\end{question}
\begin{solution}
\end{solution}

\begin{question}
真体有哪几种主要通信语言?它们各有什么特点?
\end{question}
\begin{solution}
\end{solution}

\begin{question}
多真体系统有哪几种基本模型?其体系结构又有哪几种? 
\end{question}
\begin{solution}
\end{solution}

\begin{question}
试说明多真体的协作方法、协商技术和协调方式。
\end{question}
\begin{solution}
\end{solution}

\begin{question}
为什么多真体需要学习与规划?
\end{question}
\begin{solution}
\end{solution}

\begin{question}你认为多真体系统的研究方向应是哪些?其应用前景又如何?
\end{question}
\begin{solution}
\end{solution}

\begin{question}
选择一个你熟悉的领域,编写一页程序来描述真体与环境的作用。说明环境是否是可访问的、确定性的、情节性的、静态的和连续的。对于该领域,采用何种真题结构为好?
\end{question}
\begin{solution}
\end{solution}

\begin{question}
设计并实现集中具有内部状态的真体,并测量其性能。对于给定的环境,这些真体如何接近理想的真体?
\end{question}
\begin{solution}
\end{solution}

\begin{question}
改变房间的形状和摆设物的位置,添加新家具。试测量该新环境中各真体,讨论如何改善其性能,以求处理更为复杂的地貌。
\end{question}
\begin{solution}
\end{solution}

\begin{question}
有些真体一旦得知一个新句子,就立即进行推理,而另一些真题只有在得到请求后才进行推理。这两种方法在知识层、逻辑层和执行层将有何区别?
\end{question}
\begin{solution}
\end{solution}

\begin{question}
应用布尔电路为无名普斯世界设计一个逻辑真题。该电路是一个连接输入(感知阀门)和输出(行动阀门)的逻辑门的集合。
	\begin{enumerate}
	\item 试解释为什么需要触发器;
	\item 估计需要多少逻辑门和触发器。
	\end{enumerate}
\end{question}
\begin{solution}
\end{solution}