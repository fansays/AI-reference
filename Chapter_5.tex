\chapter{计算智能}

\begin{question}
计算智能的含义是什么?它涉及哪些研究分支?
\end{question}
\begin{solution}
\end{solution}

\begin{question}
试述计算智能(CI)、人工智能(AI)和生物智能(BI)的关系。
\end{question}
\begin{solution}
\end{solution}

\begin{question}
人工神经网络为什么具有有人的发展前景和潜在的广泛应用领域?
\end{question}
\begin{solution}
\end{solution}

\begin{question}
简述生物神经元及人工神经网络的结构和主要学习算法。
\end{question}
\begin{solution}
\end{solution}

\begin{question}
考虑一个具有阶梯形阈值函数的神经网络,假设
	\begin{enumerate}
		\item 用一常数乘所有的权值和阈值;
		\item 用一常数加所有的权值和阈值。
	\end{enumerate}
试说明网络性能是否会变化?
\end{question}
\begin{solution}
\end{solution}

\begin{question}
构作一个神经网络,用于计算含有两个输入的XOR函数。指定所用神经网络单元的种类。
\end{question}
\begin{solution}
\end{solution}

\begin{question}
假定有个具有线性激励函数的神经网络,即对于每个神经元,其输出等于常数$c$乘以各输入加权和。
	\begin{enumerate}
		\item 设该网络有个隐含层。对于给定的权$W$,写出输出层单元的输出值,此值以权$W$和输入层$I$为函数,而对隐含层的输出没有任何明显的叙述。试证明:存在一个不含隐含单位的网络能够计算上述同样的函数。
		\item 对于具有任何隐含层数的网络,重复进行上述计算。从中给出线性激励函数的结论。
	\end{enumerate}
\end{question}
\begin{solution}
\end{solution}

\begin{question}
试实现一个分层前馈神经网络的数据结构,为正向评价和反向传播提供所需信息。应用这个数据结构,写出一个神经网络输出,以作为一个例子,并计算该网络适当的输出值。
\end{question}
\begin{solution}
\end{solution}

\begin{question}
什么是模糊性?它的对立含义是什么?试各举出两个例子加以说明。
\end{question}
\begin{solution}
\end{solution}

\begin{question}
什么是模糊集合和隶属函数或隶属度?
\end{question}
\begin{solution}
\end{solution}

\begin{question}
模糊集合有哪些运算?满足哪些规律?
\end{question}
\begin{solution}
\end{solution}

\begin{question}
什么是模糊推理?
\end{question}
\begin{solution}
\end{solution}

\begin{question}
对某种产品的质量进行抽查评估。现随机选出5个产品$x_1$,$x_2$,$x_3$,$x_4$,$x_5$进行检验,它们质量情况分别为:
\[ x_1=80, x_2=72, x_3=65, x_4=98, x_5=53\]
这就确定了一个模糊集合$Q$,表示该组产品的``质量水平‘’这个模糊概念的隶属程度。
试写出该模糊集。
\end{question}
\begin{solution}
\end{solution}

\begin{question}
试述遗传算法的基本原理,并说明遗传算法的求解步骤。
\end{question}
\begin{solution}
\end{solution}

\begin{question}
如何利用遗传算法求解问题,试举例说明求解过程。
\end{question}
\begin{solution}
\end{solution}

\begin{question}
用遗传算法求$f(x)=x\cos x + 2$的最大值。
\end{question}
\begin{solution}
\end{solution}

\begin{question}
什么是人工生命?请按你的理解用自己的语言给人工生命下个定义。
\end{question}
\begin{solution}
\end{solution}

\begin{question}
人工生命要模仿自然生命的特征和现象。自然生命有哪些共同特征?
\end{question}
\begin{solution}
\end{solution}

\begin{question}
为什么要研究人工生命?
\end{question}
\begin{solution}
\end{solution}

\begin{question}
人工生命包括哪些研究内容?其研究方法如何?
\end{question}
\begin{solution}
\end{solution}